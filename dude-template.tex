\usepackage{eurosym}
\usepackage[german]{babel}
\usepackage{amsmath}
\usepackage{amssymb}
\usepackage[automark]{scrpage2}	%Kopfzeile Autoinhalt (Kapitel)
\usepackage{graphics}
\usepackage{color}
\usepackage{graphicx}
\usepackage{longtable}
\usepackage{lscape}
\usepackage{hhline}
\usepackage{booktabs}
\usepackage{IFSlogo}
\usepackage{multirow}
% \usepackage[T1]{fontenc}
% \usepackage[pdftex]{hyperref}
\usepackage{makeidx}
\selectlanguage{german}
\setlength{\parindent}{0pt}  % setzt sie Einrückung nach einem Umbruch zurück

%%%%%%%%%%%%%%%%%%%%%%%%%%%%%%%%%%%%%%%%%%%%%%%%
%		Sachregister-Erstellung
%		Begriffe werden mit Befehl \index{} aufgenommen
% 		Index erstellen in Konsole makeindex -g -s style.ist Vorlage.idx 
%%%%%%%%%%%%%%%%%%%%%%%%%%%%%%%%%%%%%%%%%%%%%%%% 
\makeindex


%%%%%%%%%%%%%%%%%%%%%%%%%%%%%%%%%%%%%%%%%%%%%%%%
%		ETH-Schrift einführen
%%%%%%%%%%%%%%%%%%%%%%%%%%%%%%%%%%%%%%%%%%%%%%%% 

\usepackage[standard-baselineskips]{cmbright} % Mathematikschrift die in etwa ETH-Light entspr.
\renewcommand{\sectfont}{\bfseries}

%\renewcommand{\familydefault}{let} 
%\renewcommand{\seriesdefault}{let}
%\renewcommand{\shapedefault}{let}
%\renewcommand{\sfdefault}{let}
\renewcommand{\rmdefault}{let}


% \DeclareFixedFont{\x}{T1}{let}{m}{n}{10}
% \DeclareFixedFont{\xb}{T1}{let}{m}{n}{10}
% \newfont{\xiiiv}{letr8t at 8.0pt}
% \newfont{\xiiivb}{letb8t at 8.0pt}


%%%%%%%%%%%%%%%%%%%%%%%%
% Schriftengefrickel
%%%%%%%%%%%%%%%%%%%%%%%%%
\usepackage{fontspec}
\usepackage{sectsty}

\partfont{\font \x="DINNeuzeitGroteskStd-Light" at 40pt\x}
\chapterfont{\font \x="DINNeuzeitGroteskStd-Light" at 32pt\x}
\sectionfont{\font \x="DINNeuzeitGroteskStd-Light" at 16pt\x}
\subsectionfont{\font \x="DINNeuzeitGroteskStd-Light" at 14pt\x}
\subsubsectionfont{\font \x="DINNeuzeitGroteskStd-Light" at 14pt\x}
\paragraphfont{\font \x="DINNeuzeitGroteskStd-Light" at 14pt\x}
\newfontface\swashed[Contextuals=Swash, Ligatures=Common]{Adobe Garamond Pro Italic}
\newfontface\foo[Numbers={OldStyle},Contextuals=Swash, Ligatures=Common]{Adobe Garamond Pro}
\newfontface\foofat[Numbers={OldStyle},Contextuals=Swash, Ligatures=Common]{Adobe Garamond Pro Bold}


\usepackage[a4paper,left=4.0cm, right=4.0cm,top=3.0cm, bottom=3.0cm]{geometry}
	
%%%%%%%%%%%%%%%%%%%%%%%%%%%%%%%%%%%%%%%%%%%%%%%%
%		eigenen Stil definieren
%%%%%%%%%%%%%%%%%%%%%%%%%%%%%%%%%%%%%%%%%%%%%%%% 
\pagestyle{scrheadings}
% \renewcommand*{\chapterpagestyle}{scrheadings} 
\clearscrheadfoot 
\ihead{\textsf{\headmark}} 
\ohead{\pagemark}
\setheadsepline{.4pt}
\setfootsepline{.4pt}
\ifoot{}
\ofoot{\footnotesize{\textsc{Hsr, Silvio Heuberger}}}


% frickelabst�nde

% abst�nde und sooooo
\setlength{\columnsep}{10mm}
\setlength{\parskip}{2.5mm}

%two column float page must be 90% full
\renewcommand\dblfloatpagefraction{.90}
%two column top float can cover up to 80% of page
\renewcommand\dbltopfraction{.80}
%float page must be 90% full
\renewcommand\floatpagefraction{.90}
%top float can cover up to 80% of page
\renewcommand\topfraction{.80}
%bottom float can cover up to 80% of page
\renewcommand\bottomfraction{.80}
%at least 10% of a normal page must contain text
\renewcommand\textfraction{.1}


%%%%%%%%%%%%%%%%%%%%%%%%%%
% listings for everyone

\definecolor{defgray}{cmyk}{0.3,0.05,0,0.43}
\usepackage[plainpages={false}, bookmarks, pdfstartview={FitV}, colorlinks, linkcolor=defgray]{hyperref}


\usepackage{listings}
\lstset{% general command to set parameter(s)
basicstyle=\ttfamily\small, % print whole listing small
%basicstyle=\small,
commentstyle=\color{defgray}, % comments
stringstyle=\ttfamily, % typewriter type for strings
showstringspaces=false,
keywordstyle=\bfseries\color{blue},
numbers=left,
numberstyle=\color{defgray}\tiny,
frame=single,
%frame=shadowbox,
%frameround=tttt,
rulesepcolor=\color{defgray},
language={ruby}
} % no special string spaces


% make monospace font the same size (optocally) as Arno Pro
\setmonofont[Scale=0.9]{ITC American Typewriter Std Condensed}
\defaultfontfeatures{Scale=MatchLowercase,Mapping=tex-text}